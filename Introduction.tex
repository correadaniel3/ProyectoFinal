\chapter*{Introducción}
La  corporación   Ruta  N  es  una   entidad  pública  que pertenece  a  la  alcaldía  de  Medellín  en  conjunto  con  UNE y  EPM  la  cual  nace  en  el  año  2009. La  corporación desarrolla  distintos  programas  y  servicios  para  facilitar  la evolución económica de la ciudad hacia negocios intensivos en ciencia,  tecnología e innovación,  de forma incluyente y sostenible. Ruta N busca articular y dinamizar el ecosistema de  innovación  de  Medellín,   haciendo  énfasis  en  cuatro factores clave:  la formación del talento, el acceso a capital, la generación de la infraestructura necesaria y el desarrollo de  negocios  innovadores. Con  la  aplicación  de  estos
factores, Ruta N busca promover una cultura innovadora, la generación de empleo, el fortalecimiento de las instituciones, la formación del talento y el acceso a mercados.\\

Con  el  fin  de  cumplir  con  el  principal  propósito  de  la compañía,  mejorar la calidad de vida de los habitantes,  no solo de Medellín, si no de Colombia, Ruta N identificó diferentes  estrategias  aplicadas  alrededor  del  mundo  las  cuales se presentan como una referencia tecnológica (por ejemplo, el proyecto "Distrito22" de Barcelona aplicado desde el año 2000 en dicha ciudad).  Es así como la Corporación Ruta N, desde la Gerencia de Proyectos Especiales, pone en marcha el  proyecto  Gran  Pacto  MedellInnovation  en  septiembre
de  2014  el  cual,  posteriormente,  se  convierte  en  el  Gran Pacto por la Innovación.  Este proyecto pretende incentivar la inversión de las empresas colombianas en actividades de ciencia, tecnología e innovación.\\

El  proyecto  consta  de  tres  fases,  la  primera  de  ellas, llamada  movilización  y  firma,  consiste  en  incentivar  a  las empresas  a  evaluar  su  capacidad  de  innovación  con  una
herramienta  de  autodiagnóstico  que  provee  Ruta  N.  Al completar esta evaluación la empresa se considera firmante y  puede  acceder  a  ciertos  beneficios  como  herramientas y   asesorías   para   potenciar   o   implementar   acciones   de
innovación. Estas  acciones  están  contenidas  dentro  de la  segunda  fase  del  proyecto,   denominada  sistemas  de innovación.
Paralelamente  se  encuentra  la  tercera  fase,
llamada comunidad de innovación, en la cual se capacita a las empresas en el uso de las herramientas y se las introduce en  una  colectividad  de  compañías  donde  pueden  instaurar negocios.   Una  de  estas  herramientas  de  innovación  es  la
plataforma  SUNN  la  cual  pretende  conectar  la  oferta  de innovación  (grupos  de  investigación  y  startups)  con  su respectiva  demanda  (empresas  e  inversionistas)  girando  en torno  a  cinco  ecosistemas  de  innovación:   ciencias  de  la
vida,  materiales  avanzados,  energía,  tecnologías  limpias  y tecnologías avanzadas de la información y la comunicación.\\

Actualmente   SUNN   cuenta   en   sus   bases   de   datos, con   gran   cantidad   de   información   de   organizaciones pertenecientes a esta comunidad, entre las que se encuentran 254 startups, 361 grupos de investigación, 18 inversionistas y 1444 empresas.\\

La   información   contenida   por   la   aplicación   SUNN presenta  una  potencial  utilidad  en  el  proceso  de  toma de  decisiones  y  del  modelo  de  negocio  en  general.    Sin
embargo,  actualmente  la  aplicación  no  presenta  un  diseño arquitectónico   eficiente,   por   lo   que   con   frecuencia   se evidencian  errores  graves  en  el  funcionamiento,  lo  que  se
traduce en, una pobre experiencia de usuario, un desarrollo complejo  e  incomprensible  ya  que  ha  pasado  por  manos de  muchas  personas  sin  la  documentación  necesaria  y  una escalabilidad  nula  ocasionada  por  el  diseño  de  sus  bases de  datos  el  cual  no  cuenta  con  la  integridad  referencial adecuada.\\

Para   solventar   estos   inconvenientes   se   utilizó   una metodología  ágil  basada  en  SCRUM  con  el  fin  de  realizar entregables  periódicos,  identificar  errores  de  manera  oportuna y realizar cambios en el menor tiempo posible.

