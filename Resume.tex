\chapter*{Resumen}
La plataforma SUNN presentaba una arquitectura monolítica, un mal diseño relacional en su
base  de  datos  y  un  desarrollo  mal  estructurado  lo  que  ocasionaba  constantes  errores  en  el
funcionamiento de la misma muy evidentes en la etapa de registro de los usuarios y en las
conexiones entre actores.  Ademas se tenía desconocimiento de los procesos y criterios que
tomaba  la  aplicación  para  valorar  los  pérfiles.   En  base  a  esto  se  logró  identificar  a  fondo
el funcionamiento de las métricas de la plataforma, se construyó una arquitectura basada en
microservicios y se iniciío el desarrollo de la nueva versión de la plataforma.
